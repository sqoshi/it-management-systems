\section{Treść Projektu}

\includegraphics[width=0.9\textwidth]{tresc.pdf}\\[0.5cm]

\section{Sprawozdanie}

\subsection{Wprowadzenie}

Projekt polegał na przygotowaniu dokumentacji projektowej typu business blueprint oraz prototypowego wdrożenia systemu klasy ERP dla średniego przedsiębiorstwa produkcyjno-usługowego. W ramach projektu stworzono szereg modeli i diagramów, przeprowadzono konfigurację systemu Odoo Community, oraz zrealizowano zadanie optymalizacji procesu biznesowego.

\subsection{Przegląd działań projektowych}

Organizacja i modele procesów:
Przygotowano modele podstawowych procesów biznesowych firmy, zdefiniowano dane podstawowe oraz strategię wdrażania systemów. Modele te stworzone zostały za pomocą narzędzia Bizagi\citep{bizagi}, co umożliwiło dokładne odwzorowanie procesów z uwzględnieniem dostawców i klientów.

Instalacja i konfiguracja ERP:
System Odoo Community\citep{odoo} został zainstalowany, skonfigurowany i przygotowany do prezentacji. Wprowadzono przykładowe rekordy produktów, klientów oraz dokumentów, które ukazują różne aspekty funkcjonowania systemu. Proces konfiguracji był realizowany współbieżnie z projektowaniem procesów biznesowych, co pozwoliło na bieżące dostosowywanie systemu do potrzeb przedsiębiorstwa.

Optymalizacja procesu biznesowego:
Wybrany fragment procesu biznesowego został zoptymalizowany za pomocą języka notacji AMPL\cite{ampl}. Model przetestowano na stronie internetowej AMPL oraz w narzędziu GNU MathProg, co umożliwiło sprawdzenie poprawności modelu i znalezienie optymalnych rozwiązań.

\subsection{Refleksja dotycząca pozyskanej wiedzy i umiejętności}

Rozszerzenie wiedzy i umiejętności:
Projekt pozwolił na znaczne poszerzenie wiedzy z zakresu projektowania systemów ERP oraz modelowania procesów biznesowych. Dzięki pracy z Odoo Community zdobyto praktyczne umiejętności związane z konfiguracją i wdrażaniem systemów ERP. Dodatkowo, wykorzystanie narzędzia Bizagi do tworzenia diagramów BPMN przyczyniło się do lepszego zrozumienia i wizualizacji procesów biznesowych.

Materiały bazowe:
Bazą do realizacji projektu były wykłady z przedmiotu „Wspomaganie decyzji” oraz dokumentacja dostępna na stronach internetowych, takich jak dokumentacja oprogramowania Odoo. Ponadto, wykorzystano materiały dotyczące języka AMPL oraz narzędzia GNU MathProg, co umożliwiło przeprowadzenie optymalizacji procesów.

Wnioski:
Zrealizowany projekt pokazał, jak istotne jest kompleksowe podejście do wdrażania systemów ERP, obejmujące zarówno modelowanie procesów biznesowych, jak i praktyczną konfigurację systemu. Pozyskana wiedza i umiejętności z zakresu narzędzi takich jak Odoo, Bizagi oraz AMPL stanowią cenne doświadczenie, które można wykorzystać w przyszłych projektach związanych z optymalizacją i wdrażaniem systemów informatycznych w przedsiębiorstwach.

\subsection{Podsumowanie}

Projekt ISZ (2024) zakończył się sukcesem, osiągając wszystkie zakładane cele. Stworzono dokumentację typu business blueprint, przeprowadzono konfigurację systemu ERP, oraz zrealizowano zadanie optymalizacji procesu biznesowego. Zdobyte doświadczenia i wiedza przyczynią się do dalszego rozwoju umiejętności w zakresie zarządzania procesami biznesowymi i wdrażania systemów informatycznych.

